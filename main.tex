\documentclass[polish,envcountsect,10pt]{beamer}
    \usepackage[T1]{fontenc}
    \usepackage{fontspec}                 % żeby ustawić czcionkę na systemową (Arial)
    %\usepackage[utf8]{inputenc}
    \usepackage{polski}
    \usepackage{babel}

    \usetheme{Madrid}

    \title{Rozwiązywanie gier}
    \author{Michał Krakowiak}
    \date{Gdańsk, \texorpdfstring{$2019$}{2019}}

\begin{document}
    \frame{\titlepage}
    \section{Wstęp}
        \subsection{Zawartość}
            \begin{frame}
                \frametitle{Zawartość}
                \tableofcontents[pausesections]
            \end{frame}
        \subsection{Definicja gry}
            \begin{frame}
                \frametitle{Definicja gry}
            \end{frame}
    \section{Klasyfikacja rozwiązań}
        \subsection{Przegląd}
            \begin{frame}
                \frametitle{Klasyfikacja rozwiązań}
                \begin{itemize}
                    \item<1-> Ultra słabe
                    \item<2-> Słabe
                    \item<3-> Mocne
                \end{itemize}
            \end{frame}
        \subsection{Ultra słabe}
            \begin{frame}
                \frametitle{Rozwiązania ultra słabe}
            \end{frame}
        \subsection{Słabe}
            \begin{frame}
                \frametitle{Rozwiązania słabe}            
            \end{frame}
        \subsection{Mocne}
            \begin{frame}
                \frametitle{Rozwiązania mocne}
                Jest znany algorytm pozwalający uzyskać najlepsze ruchy z każdej pozycji (nawet jeżeli, któryś z graczy popełnił błąd)
            \end{frame}
            \begin{frame}
                \frametitle{Kółko i krzyżyk}
            \end{frame}
    \section{Przykłady gier o znanych rozwiązaniach}
            \subsection{Kółko i krzyżyk}
            \subsection{Nim}
            \subsection{Szachy na małych planszach}
            \subsection{Warcaby}
    \section{Szachy}
        \begin{frame}
            \frametitle{Szachy}
        \end{frame}
        \subsection{Problematyka}
            \begin{frame}
               \frametitle{} 
            \end{frame}
        \subsection{Minimax}
            \begin{frame}
                
            \end{frame}
        \subsection{NegaMax}
            \begin{frame}
                
            \end{frame}
        \subsection{Alfa-Beta Pruning}
            \begin{frame}
            \end{frame}
    \section{Literatura}
        \begin{frame}
            \begin{thebibliography}{9}
                \bibitem{PLACEHOLDER}PLACEHOLDER
            \end{thebibliography}
        \end{frame}
\end{document}